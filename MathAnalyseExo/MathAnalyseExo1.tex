\documentclass[12pt]{report}

\usepackage[utf8]{inputenc}
\usepackage[T1]{fontenc}
\usepackage[french]{babel}
\usepackage{fancyhdr}

\usepackage{amsmath} % caligraphiques
\usepackage{amssymb} % les symboles R, N 
\usepackage{mathrsfs} % fancy script font
\usepackage{mathptmx} % polices
\usepackage{fancybox} %encadrer les formules
\usepackage{tikz,tkz-tab} % tableaux de signe
\usepackage{setspace} % mise en page
\usepackage{stmaryrd}
\usepackage[top=2cm, bottom=2cm, left=1cm, right=1cm]{geometry}

\newcommand{\p}[1][6]{\\[#1pt]}
\newcommand{\pt}[1]{\\[#1pt]}

\newcommand{\C}{\mathbb{C}}
\newcommand{\R}{\mathbb{R}}
\newcommand{\Q}{\mathbb{Q}}
\newcommand{\Z}{\mathbb{Z}}
\newcommand{\N}{\mathbb{N}}
\setlength{\parindent}{0 pt} 

\pagestyle{empty}
\pagestyle{fancy}
\fancyhead{} % clear all header fields
\fancyhead[CO,LE]{\textbf{Math Analyse Exo1 Equivalent de suites}}
\fancyfoot{} % clear all footer fields

\begin{document}
Soit  $(a_n)$ et $(b_n)$ deux suites réelles, montrer que
\[e^{a_n} \sim \left (1+\dfrac{a_n}{n}\right )^n \iff a_n=o(\sqrt{n})\]
\newpage
Tout d'abord, nous pouvons remarquer que, pour deux suites $(a_n)$ et $(b_n)$\p
$e^{an}\sim e^{b_n} \iff \dfrac{e^{a_n}}{e^{b_n}}\rightarrow 1 \iff e^{(a_n-b_n)} \rightarrow 1 \iff a_n-b_n \rightarrow 0$\p
Nous allons utiliser ce résultat par la suite.\p
\p
Supposons maintenant que $a_n=o(\sqrt{n})$ et montrons que $e^{a_n} \sim \left (1+\dfrac{a_n}{n}\right )^n$\p
Comme $a_n=o(\sqrt{n})$, alors $\dfrac{a_n}{n}=o(\dfrac1{\sqrt{n}})$ et de ce fait, 
\[\lim_{n \to \infty}\dfrac{a_n}{n} = 0\]
Nous pouvons faire un développement limité:
\begin{align*}
	a_n-n\ln(1+\dfrac{a_n}{n})&=a_n-n\left (\frac{a_n}{n}-\dfrac{a^2_n}{2n^2} +o(1/n)\right ) \\		a_n-n\ln(1+\dfrac{a_n}{n})&=a_n-a_n+\dfrac{a_n^2}{2n}+o(1) \\
	a_n-n\ln(1+\dfrac{a_n}{n})&=\dfrac{a_n^2}{2n}+o(1) 
\end{align*}
Et comme $\displaystyle \lim_{n \rightarrow + \infty} \dfrac{a_n^2}{2n}=0$, en utilisant le résultat décrit en introduction
\[\boxed{e^{a_n} \sim \left (1+\dfrac{a_n}{n}\right )^n} \]
\p
Pour la réciproque, supposons que $e^{a_n} \sim \left (1+\dfrac{a_n}{n}\right )^n$\p
Ce qui est équivalent à $ a_n-n\ln(1+\frac{a_n}{n})\rightarrow 0 $\p
Nous posons la fonction $f$ qui à tout $x$ de $]-1; +\infty[$ associe $f(x)=x-\ln(1+x)$\p
Une étude de cette fonction montre qu'elle est continue, admet un minimum en $0$ qui vaut $0$.\p
Comme $\displaystyle \lim_{n \to \infty}f(\dfrac{a_n}{n}) = 0$, alors nous pouvons montrer par l'absurde que $\displaystyle \lim_{n \to \infty}\dfrac{a_n}{n} = 0$\p
Supposons que la suite $(\dfrac{a_n}{n})$ ne tende pas vers $0$, alors\p
$(\exists \epsilon >0) \ (\forall n_0 \in \N) \ (n \geq n_0)=> (|\dfrac{a_n}{n}|\geq\epsilon)$\p
$(\exists \epsilon' >0) \ (\forall n_0 \in \N) \ (n \geq n_0)=> (|f(\dfrac{a_n}{n})|\geq\epsilon'>0)$\p
et alors $n\left (\dfrac{a_n}{n}-\ln(1+\dfrac{a_n}{n})\right )$ ne peut pas tendre vers $0$, ce qui contredit l'hypothèse.\p
Nous venons de démontrer par l'absurde que $\dfrac{a_n}{n}\rightarrow 0$ et donc nous pouvons faire le développement limité.

	\[a_n-n\ln(1+\dfrac{a_n}{n})=\dfrac{a_n^2}{2n} +o(1)\]

Donc forcément, $\dfrac{a_n^2}{n} \rightarrow 0$, ce qui est équivalent à $a_n=o(\sqrt{n})$, d'où le résultat.\p

\end{document}
 
